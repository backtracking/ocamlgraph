\documentclass{beamer}

\usetheme{Madrid}
%\usetheme{Marburg}
%\usecolortheme{beaver}
\usepackage{listings}

\begin{document}

\title{A Formal Proof of Bellman-Ford Algorithm}
\author[Yuto Takei]{Yuto Takei}
\institute{The University of Tokyo}

\maketitle

\begin{frame}
  \tableofcontents
\end{frame}

\section{Bellman-Ford algorithm}
% 5 to 10 slides

\begin{frame}\frametitle{Bellman-Ford algorithm}
  % explain what it does informally

  % show pseudo-code

  % run it on an example
\end{frame}

\begin{frame}\frametitle{}

\end{frame}

\begin{frame}\frametitle{}

\end{frame}

\section{Implementation in OCamlgraph}
% up to 5 slides
  % focus on the API
  % incremental version as well

\begin{frame}\frametitle{Implementation in OCamlgraph}
\end{frame}

\begin{frame}[fragile]\frametitle{}
  \begin{lstlisting}[language=Caml]
  exception NegativeCycle of G.E.t list
 
  val all_shortest_paths : G.t -> G.V.t -> W.t H.t
  \end{lstlisting}
\end{frame}


\section{Formal proof with Why3}
% 10 slides

%\subsection{Specification}

% definitions (graphs, paths, for the specification)
% + program specification (pre/post)

\begin{frame}\frametitle{}
\end{frame}

%\subsection{Proof}

% code
% additional definitions/lemmas for the proof
% additional annotations for the proof
% loop invariants

\begin{frame}\frametitle{}
\end{frame}


\section{Conclusion and Perspectives}

% what's next, still to be done, etc.

\end{document}
